\documentclass{article}

\usepackage[utf8x]{inputenc}
\usepackage[russian]{babel}

\begin{document}
  % Введение
  %Глава 1. Теоретическое обоснование дипломного проекта.
    %Анализ. Классификация. Сравнение.
    %Анализ предметной области. Т.е. компьютерной сети.
    %Рассмотрение различных способов моделирования. Обоснование выбора используемого.

    %План.
      %Введение.
      %Моделирование
    Классификация видов моделирования.

    Классификационные признаки. В качестве одного из первых признаков классификации видов моделирования можно выбрать степень полноты модели и разделить модели в соответствии с этим на полные, неполные и приближенные. В основе полного моделирования лежит полное подобие, которо проявляется как во времени, так и в пространстве. Для неполного моделирования характерно неполное подобие модели изучаемому объекту. В основе приближенного моделирования лежит приближенное подобие, при котором некоторые стороны функционирования реального объекта не моделируются совсем. Классификация видов моделирования приведена на рисунке 1.

    В зависимости от характера изучаемых процессов в системе S все виды моделирования могут быть разделены на детерминированные и стохастические, статические и динамические, дискретные, непрерывные и дискретно-непрерывные. Детерминированное моделирование отображает детерминированные процессы, т.е. процессы, в которых предполагается отсутствие всяких случайных воздействий; стохастическое моделирования отображает вероятностные процессы и события. В этом случае анализируется ряд реализаций случайного процесса и оцениваются средние характеристики, т.е. набор однородных реализаций. Статическое моделирования служит для описания поведения объекта в какой-либо момент времени, ф динамическое моделирование отражает поведение объекта во времени. Дискретное моделирование служит для описания процессов, которые предполагаются дискретными, соответственно непрерывное моделирование используется для случаев, когда хотят выделить наличие как дискретных, так и непрерывных процессов.

    В зависимости от формы представления объекта (системы S) можно выделить мысленное и реальное моделирование.

    Мысленное моделирование часто является единственным способом моделирования объектов, которые либо практически нереализуемы а заданном интервале времени, либо существуют вне условий, возможных для их физического создания. Мысленное моделирование может быть реализовано в виде наглядного, символического и математического.

    При наглядном моделировании на базе представлений человека о реальных объектах создаются различные наглядные модели, отображающие явления и процессы, протекающие в объекте. В основу гипотетического моделирования исследователем закладывается некоторая гипотеза о закономерностях протекания процесса в реальном объекте, которая отражает уровень знаний исследователя об объекте и базируется на причинно-следственных связях между входом и выходом изучаемого объекта. Гипотетическое моделирование используется, когда знаний об объекте недостаточно для построения формальных моделей.

    Аналоговое моделирование основывается на применении аналогий различных уровней. Наивысшим уровнем является полная аналогия, имеющая место только для достаточно простых объектов. С усложнением объекта используют аналогии последующих уровней, когда аналоговая модель обображает несколько либо только одну сторону функционирования объекта.

    Существенное место при мысленном наглядном моделировании занимает макетирование. Мысленный макет может применяться в случаях, когда протекающие в реальном объекте процессы не поддаются физическому моделированию, либо может предшествовать проведению других видов моделирования. В основе построения мысленных макетов также лежат аналогии, однако обычно базирующиеся на причинно-следственных связях между явлениями и процессами в объекте. Если ввести условное обозначение отдельных понятий, т.е. знаки, а также определенные операции между этими знаками, то можно реализовать знаковое моделирование и с помощью знаков отображать набор понятий -- составлять отдельные цепочки из слов и предложений.

    В основе языкового моделирования лежит некоторый тезаурус. Последний образуется из набора входящих понятий, причем этот набор должен быть фиксированным. Следует отметить, что между тезаурусом и обычным словарем имеются принципиальные различия. Тезаурус - словарь, который очищен от неоднозначности, т.е. в нем каждому слову может соответствовать лишь единственное понятие, хотя в обычном словаре одному слову могут соответствовать несколько понятий.

    Символическое моделирование представляет собой искусственный процесс создания логического объекта, который замещает реальный и выражает основные свойства его отношений с помощью определенной системы знаков или символов.

    Математическое моделирование. Для исследования характеристик процесса функционирования любой системы S математическими методами, включая и машинные, должна быть проведена формализация этого процесса, т.е. построена математическая модель.

    Под математическим моделированием понимают процесс установления соответствия данному реальному объекту некоторого математического объекта, называемого математической моделью, и исследование этой модели, позволяющее получать характеристики рассматриваемого реального объекта. Вид математического модели зависит как от природы реального объекта, так и задач исследования объекта и требуемой достоверности и точности решения этой задачи. Любая математическая модель, как и всякая другая описывает реальный объект лишь с некоторой степенью прближения к действительности. Математическое моделирование для исследвания характеристик процесса функционирования систем можно разделить на аналитическое, имитационное и комбинированное.

    Для аналитического моделирования характерно то, что процессы функционирования элементов системы записываются в виде некоторых функциональных соотношений (алгебраических, интегродифференциальных, конечно-разностных и.т.п.) или логических условий. Аналитическая модель может быть исследована следующими методами:
\begin{enumerate}
  \item аналитическим, когда стремятся получить в общем виде явные зависимости для искомых характеристик;
  \item численным, когда, не умея решать уравнений в общем виде, стремятся получить числовые результаты при конкретных начальных данных;
  \item качественным, когда, не имея решения в явном виде, можно найти некоторые свойства решения.
\end{enumerate}

    Наиболее полное исследование процесса функционирования системы можно провести, если известны явные зависимости, связывающие искомые характеристики с начальными условиями, параметрами и переменными системы S. Однако такие зависимости удается получить только лоя сравнительно простых систем. При усложнении систем исследование их аналитическим методом наталкивается на значительные трудности, которые часто бывают непреодолимыми. Поэтому, желая использовать аналитический метод, в этом случае идут на существенное упрощение первоначальной модели, чтобы иметь возможность изучить хотя бы общие свойства системы. Такое исследование на упрощенной модели аналитическим методом помогает получить ориентировочные результаты для определения более точных оценок другими методами. Численный метод позволяет исследовать по сравнению с аналитическим методом более широкий класс систем, но при этом полученные решения носят частный характер. Численный метод особенно эффективен при использовании ЭВМ.

    В отдельных случаях исследования системы могут удовлетворять и те выводы, которые можно сделать про использовании качественного метода анализа математической модели. Такие качественные методы широко используются, например, в теории автоматического управления для оценки эффективности различных вариантов систем управления.

    В настоящее время распространены методы машинной реализации исследования характеристик процесс функционирования больших систем. Для реализации математической модели на ЭВМ необходимо построить соответствующий моделирующий алгоритм.

    При имитационном моделировании реализующий модель алгоритм воспроизводит процесс функционирования системы S во времени, причем имитируются элементарные явления, составляющие процесс, с сохранением их логической структуры и последовательности протекания во времени , что позволяет по исходным данным получить сведения о состояниях процесса в определенные моменты времени, дающие возможность оценить характеристики системы S.

    Основным преимуществом имитационного моделирования по сравнению с аналитическим является возможность решения более сложных задач. Имитационные модели позволяют достаточно просто учитывать такие факторы, как наличие дискретных и непрерывных элементов, нелинейные характеристики элементов системы, многочисленные случайные воздействия и др., котоые часто создают трудности при аналитических исследованиях. В настоящее время имитационное моделирование -- наиболее эффективный метод исследования больших систем, а часто и единственный практически доступный метод получения информации о поведении системы, особенно на этапе ее проектирования.
    Когда результаты, полученные про воспроизведении на имитационной модели процесса функционирования системы S, являются реализациями случайных величин и функций, тогда для нахождения характеристик процесса требуется его многократное воспроизведение с последующей статистической обработкой информации и целесообразно в качестве метода машинной реализации имитационной модели использовать метод статистического моделирования. Первоначально был разработан метод статистических испытаний, представляющий собой численный метод, который применялся для моделирования случайных величин и функций, вероятностные характеристики которых совпадали с решениями аналитических задач (такая процедура получила название метода Монте-Карло). Затем этот прием стали применять и для машинной имитации с целью исследования характеристик процессов функционирования систем, подверженных случайным воздействиям, т.е. появился метод статистического моделирования. Таким образом, методом статистического моделирования называют метод машинной реализации имитационной модели, а методом статистических испытаний -- численный метод решения аналитической задачи.

    Метод имитационного моделирования позволяет решать задачи анализа больших систем S, включая задачи оценки: вариантов структуры систем, эффективности различных алгоритмов управления системой, влияния изменения различных параметров системы. Имитационное моделирование может быть положено также в основу структурного. алгоритмического и параметрического синтеза больших систем, когда требуется создать систему, с заданными характеристиками при определенных ограничениях, которая является оптимальной по некоторым критериям оценки эффективности.

    При решении задач машинного синтеза систем на основе их имитационных моделей помимо разработки моделирующих алгоритмов для анализа фиксированной системы необходимо так же разработать алгоритмы поиска оптимального варианта системы. Далее в методологии машинного моделирования будем различать два основных раздела: статику и динамику, -- основным содержанием которых являются соответственно вопросы анализа и синтеза систем, заданных моделирующими алгоритмами.

    Комбинированное (аналитико-имитационное) моделирование при анализе и синтезе систем позволяет объединить достоинства аналитического и имитационного моделирования. При построении комбинированных моделей проводится предварительная декомпозиция процесса функционирования объекта на составляющие подпроцессы и для тех из них, где это возможно, используются аналитические модели, а для остальных подпроцессов строятся имитационные модели. Такой комбинированный подход позволяет охватить качественно новые классы систем, которые не могут быть исследованы с использованием только аналитического и имитационного моделирования в отдельности.

    Другие виды моделирования.

    При реальном моделировании используется возможность исследования различных характеристик либо на реальном объекте целиком, либо на его части. Такие исследования могут проводиться как на объектах, работающих в нормальных режимах, так и при организации специальных режимов, для оценки интересующих исследователя характеристик. Реальное моделирование является наиболее адекватным, но при этом его возможности с учетом особенностей реальных объектов ограничены.

    Натурным моделированием называют проведение исследования на реальном объекте с последующей обработкой результатов эксперимента на основе теории подобия. При функционировании объекта в соответствии с поставленной целью удается выявить закономерности протекания реального процесса. Надо отметить, что такие разновидности натурного эксперимента, как производственный эксперимент и комплексные испытания, обладают высокой степенью достоверности.

    Другим видом реального моделирования является физическое, отличающееся от натурного тем, что исследование проводится на установках, которые сохраняют природу явлений и обладают физическим подобием. А процессе физического моделирования задаются некоторые характеристики внешней среды и исследуется поведение либо реального объекта, либо его модели при заданных или создаваемых искусственно воздействиях внешней среды. Физическое моделирование может протекать в реальном и нереальном (псевдореальном) масштабах времени, а так же рассматриваться без учета времени. В последнем случае изучению подлежат так называемые "Замороженные" процессы, которые фиксируются в некоторый момент времени.

        %Описание способов моделирования процессов
        %Сравнительный анализ описанных способов.
        %Системы моделирования процессов. Сложность использования для задачи.
        %Почему не использую языки моделирования
        %Почему не использую готовые решения.
      %Безопасность
        %Возможные способы нарушения безопасности с использованием локальной сети.
        %Подходы к анализу безопасности в локальных сетях.
        %Описание различных моделей для оценки безопасности.
        %Сравнительный анализ описанных моделей.
      %Итог. Выбор способа моделирования. Описание требований.

  %Глава 2.
    %Принципы построения модели
    %Используемые алгоритмы
    %Теория подобия
  %Глава 3. Софт
  %Глава 4. Экономическая оценка.
  %Глава 5. Правовая оценка.
  %Заключение.
  %Список литературы.

\end{document} 