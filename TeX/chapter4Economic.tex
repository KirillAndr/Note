
\newpage

\chapter{Экономическая оценка}
\section{Концепция экономической оценки}


В современном мире невозможно представить себе область деятельности, в которой не используются средства вычислительной техники. Процесс обработки информации с использованием таких средств, становится в большей степени автоматизированным, что позволяет значительно увеличить эффективность работы. Для обеспечения удобного обмена данными между локальными рабочими станциями, передачи данных в хранилище и.т.д. их необходимо объединить в сеть.  При этом обрабатываемые данные могут быть различной степени конфиденциальности, кроме того, вопросы целостности и доступности имеют большое значение в процессе эффективного функционирования организации. Однако, объединение средств вычислительной техники в сети открывает некоторые возможности для злоумышленника получить доступ к данным или  иным способом нарушить информационный обмен между рабочими станциями.

Нарушение нормальной работоспособности сети может повлечь за собой неприемлемые для организации последствия. Например, фирма может потерять прибыль из-за нарушения конфиденциальности данных или неспособности оказать услугу в данный момент времени, из-за нарушения стабильной работы сервера.
Для сокращения риска возникновения подобных ситуаций, организации используют различные средства защиты. Это могут быть как аппаратные средства, такие как, например,  межсетевые экраны, программные средства, такие как антивирусы, системы анализа трафика или комплексные решения, например, системы обнаружения вторжения.

Эффективность использования упомянутых средств зависит не только от их качества, но и от правильного их использования, что является задачей инженера по безопасности. Однако сложность правильного использования средств защиты напрямую зависит от сложности сети. В случае больших сетей, которые имеют место в крупных организациях, упомянутая сложность становится не только угрозой безопасности, но проблемой неэффективного использования финансовых средств.
Для проведения предварительного анализа и выявления возможных проблем используются различные системы моделирования. Существуют системы, решающие проблему доступности данных, т.е. позволяющих построить в заданных условиях наиболее быстро функционирующую сеть. Другие системы, такие как сканеры уязвимостей, позволяют выявлять наличие мест, в которых потенциально возможна успешная атака на защищаемые данные. Некоторые системы позволяют оценить возможность той или иной атаки на сеть.

Темой дипломного проекта является: "Разработка системы моделирования атак на вычислительную сеть". Разрабатываемая система обладает рядом преимуществ:
\begin{enumerate}
\item возможность оценки вероятности компрометации какой-либо системы в составе локальной сети.
\item возможность оценки состояния сети при атаках типа "отказ в обслуживании".
\item удобный пользовательский интерфейс.
\end{enumerate}

Данная система позволяет инженеру по безопасности провести моделирование различных режимов функционирования сети, в т.ч.  режимы атаки, и по результатам тестирования принять решение о необходимости использования тех или иных средств защиты или изменения конфигурации сети. Разрабатываемая система объединяет в себе возможности моделирования потоков сетевого трафика и возможностей по эксплуатации уязвимостей. Таким образом, данная система позволяет уменьшить риски при работе с конфиденциальной информацией.

Данная разработка является независимой и автономной.

Разрабатываемая система может быть применена при проектировании архитектуры крупных вычислительных сетей с целью анализа состояний объектов, участвующих в информационном обмене и выявлении возможных точек образования очередей и неисправностей.

Экономическое обоснование не производится, а производится ее оценка, т.к. на данном этапе НИР отсутствует информация о результатах применения.

Экономическая оценка эффективности проводимой в дипломном проекте научно-исследовательской работы направлена на доказательство актуальности тематики, а также на оценку научно-технического уровня полученных результатов. Экономическая оценка работы будет состоять из следующих разделов:

\begin{enumerate}
\item трудоемкость выполнения НИР.
\item смета затрат на проведение НИР
\item экономическая оценка эффективности НИР
\end{enumerate}

Выполнение дипломной работы проходило на базе Санкт-Петербургского государственного электротехнического университета. 

\section{Трудоемкость выполнения НИР}
В данном разделе рассмотрим перечень основных этапов и видов работ, которые должны быть выполнены для достижения целей НИР.

Данные по трудоемкости и срокам выполнения НИР приведены в таблице~\ref{table1}.

Исполнителями НИР являются старший научный сотрудник и студент-выпускник высшего учебного заведения (инженер). Учитывая эти данные, была определена трудоемкость выполняемых работ.

\begin{table}
\caption{Трудоемкость и сроки выполнения НИР.}\label{table1}
\begin{tabular}{|p{1em}|p{15em}|p{7em}|p{6em}|p{8em}|}
  \hline
  № & Наименование работ & \multicolumn{2}{|c|}{Трудоемкость, чел/дни} & Машинное время маш/час \\
  \ & \ & Старший научный сотрудник & Инженер & \ \\ \hline
  1 & Постановка задачи и утверждение ТЗ. & 1 & 1 & - \\ \hline
  2 & Сбор информации по существующим решениям и анализ полученных данных & - & 5 & 30 \\ \hline
  3 & Анализ научных работ в исследуемой области & - & 5 & 20 \\ \hline
  4 & Анализ методов моделирования. & - & 5 & 30  \\ \hline
  5 & Формулировка технического задания  & 1 & 1 & - \\ \hline
  6 & Разработка моделей устройств для системы  & 2 & 10 & 60 \\ \hline
  7 & Разработка архитектуры приложения & 2 & 10 & 60  \\ \hline
  8 & Реализация системы & 1 & 20 & 150  \\ \hline
  9 & Анализ и оценка разработанной системы & 1 & 1 & 5 \\ \hline
  10 & Исправление ошибок & - & 5 & 30  \\ \hline
  11 & Составление и оформление отчета по НИР & - & 10 & 40 \\ \hline
  12 & Сдача проекта & 1 & 1 & - \\ \hline
  \multicolumn{2}{|r|}{ИТОГО:} & 9 & 73 & 435 \\ \hline
\end{tabular}
\end{table}

\section{Смета затрат на проведение НИР}
В этом разделе мы оценим затраты, необходимые на проведение НИР. Основными статьями калькуляции являются материалы, спецоборудование, расходы на оплату труда, отчисления на социальные нужды, прочие прямые расходы и накладные расходы.

\subsubsection{Статья "Материалы"}

К статье "Материалы" отнесем расходы, представленные в таблице~\ref{table2}.

\begin{table}[h!]\center
\caption{Расходы на материалы}\label{table2}
\begin{tabular}{|p{10em}|p{10em}|p{9em}|p{9em}|}
  \hline
  Наименование товара & Количество & Стоимость за единицу, руб & Стоимость, руб \\ \hline
  Картридж для принтера & 1 & 2500 & 2500 \\ \hline
  Бумага А4 & 1 пачка & 150 & 150 \\ \hline
  Канцелярские принадлежности & 1 комплект & 100 & 100 \\ \hline
  \multicolumn{3}{|c|}{Итого} & 2750 \\ \hline
  \multicolumn{3}{|c|}{Транспортно-заготовительные расходы(15\%)} & 410 \\ \hline
  \multicolumn{3}{|c|}{Итого} & 3160 \\ \hline
\end{tabular}
\end{table}

\subsubsection{Статья "Спецоборудование"}
  Расходы на данную статью не предусмотрены.

\subsubsection{Статья "Расходы на оплату труда"}

Данные по статье "Расходы на оплату труда" представлены в таблице~\ref{table3}.
\begin{table}[h!]
\caption{Расходы на оплату труда}\label{table3}
\begin{tabular}{|p{12em}|p{10em}|p{9em}|p{7em}|}
  \hline
  Участник НИР & Зар.плата за месяц & Количество рабочих дней & Общая сумма \\ \hline
  Старший научный сотрудник & 40000 & 9 & 19200 \\ \hline
  Инженер & 25000 & 73 & 97300 \\ \hline
  \multicolumn{3}{|c|}{Итого} & 116500 \\ \hline
\end{tabular}
\end{table}

Основная заработная плата исполнителей рассчитывается по формуле:

\begin{center}
$C_{ЗО} = \dfrac{T_{1} \cdot C_{ЗОмес}}{t} \cdot (1 + \dfrac{H_{H}}{100})$
\end{center}

где $T_{1}$ -- трудоемкость выполнения работ руководителя и инженера.

    $C_{\text{ЗОмес}}$ -- месячные оклады исполнителей.

    $H_{H}$ -- норматив начислений, 12\%

    $t$ -- среднее количество рабочих дней в месяце(21).

\subsubsection{Статья "Страховые взносы в государственные внебюджетные фонды"}

Ставка социального налога составляет 30\%, включающие в себя отчисления в:

\begin{enumerate}
  \item Пенсионный фонд Российской Федерации -- 22\%
  \item Фонд Социального страхования Российской Федерации -- 2,9\%
  \item Федеральный фонд обязательного медицинского страхования -- 2,1\%

\end{enumerate}
Таким образом налог составляет:

  $C_{\text{CH}} = C_{\text{ЗО}} \cdot \dfrac{H_{CH}}{100}$

  $C_{\text{CH}} = 116500 \cdot \dfrac{30}{100} = 34950$

\subsubsection{Статья "Затраты по работам, выполненным сторонними организациями"}

В качестве расходов на оплату услуг сторонних организаций условно выступает стоимость машинного времени.

Стоимость машинного времени рассчитывается по формуле:

\begin{center}
$C_{\text{МВ}} = t_{\text{МВ}} \cdot P_{\text{МВ}}$
\end{center}

где:

$t_{\text{МВ}}$ -- время использования ПЭВМ(435 часов)

$P_{\text{МВ}}$ -- стоимость машиннго часа времени (условно 30 руб/час)

$C_{\text{МВ}} = 435 \cdot 30 = 13050$ руб.

\subsubsection{Статья "Командировочные расходы"}
Затраты на служебные командировки не предусмотрены.

\subsubsection{Статья "Прочие прямые расходы"}

К статье "прочие прямые расходы" относятся расходы на получение специальной научно-технической информации, за использование средств связи и коммуникации и другие расходы, необходимые для проведения НИР.

В таблице~\ref{table4} представлены статьи, относящиеся к прочим прямым расходам.

\begin{center}
\begin{table}\center
\caption{Прочие прямые расходы}\label{table4}
\begin{tabular}{|c|c|}
  \hline
  Наименование & Сумма, руб. \\ \hline
  Internet & 300 \\ \hline
  Книги & 1500  \\ \hline
  Другие расходы &  216 \\ \hline
  Итого & 2016\\ \hline
\end{tabular}
\end{table}


$C_{\text{ПР}} = C \cdot k$

$C_{\text{ПР}} = 1800 \cdot 1,12$ = 2016

\end{center}

где k -- коэффициент учитывающий другие виды прочих прямых расходов.

\subsubsection*{Статья "Накладные расходы"}

В статью "накладные расходы" мы включим все расходы на управление и хозяйственное обслуживание. Так как в период проведения НИР в работе оборудования не было никаких сбоев, то затраты, связанные с его ремонтом не учитываем. Величина накладных расходов определяется на основании норматива, установленного в СПбГЭТУ, и берется равной 33\%.

\begin{center}
$C_{\text{НР}} = C_{\text{ЗО}} \cdot \frac{H_{HP}}{100}$

$C_{\text{НР}} = 116500 \cdot \frac{33}{100} = 38445$
\end{center}

\subsubsection*{Статья "Себестоимость НТПр"}
В таблице~\ref{table5} представлены все основные статьи калькуляции расходов, необходимых для проведения НИР.

Себестоимость рассчитывается по формуле:

$C = C_{M} + C_{\text{ЗО}} + C_{CH} + C_{\text{МВ}} + C_{\text{ППР}} + C_{HP} = 208121$

\begin{table}[h!]
\caption{Себестоимость НТПр}\label{table5}
\begin{tabular}{|c|c|}
  \hline
  Статья затрат & Сумма, руб \\ \hline
  Материалы & 3160 \\ \hline
  Спецоборудование &  - \\ \hline
  Расходы на оплату труда & 116500\\ \hline
  Страховые взносы в государственные внебюджетные фонды & 34950\\ \hline
  Затраты по работам, выполненным сторонними организациями & 13050\\ \hline
  Командировочные расходы & - \\ \hline
  Прочие прямые расходы & 2016\\ \hline
  Накладные расходы & 38445\\ \hline
  Итого Себестоимость & 208121\\ \hline
\end{tabular}
\end{table}

\textbf{Итого общая стоимость разработки составляет $C_{o} = 208121$ руб.}

\section{Комплексная оценка эффективности НИР.}

В данной разработке производится качественная оценка экономической эффективности НИР. Это объясняется этапом выполнения НИР, а также отсутствием на данном этапе информации о применении разработки и возможности ее сбыта, потому оценка экономической эффективности НИР включает качественную оценку эффекта от разработки и научно-технические уровни (уровень качества) выполненной разработки.

Количественная оценка экономической эффективности НИР отсутствует, в связи с тем, что на данном этапе отсутствует информация о результатах применения разработки.

Разработанная система предназначена для решения ряда задач, в том числе таких, для решение которых не предусмотрено в иных программных комплексах, близких по назначению к данной системе.  Произведем оценку уровня качества в сравнение с системой Cisco Packet Tracer(П1)(таблица~\ref{table6}).

\begin{table}[h!]
\caption{Сравнение характеристик конкурирующей разработки} \label{table6} 
\begin{tabular}{|p{1em}|p{17em}|p{4em}|p{7em}|p{7em}|}
  \hline
  № & Характеристики & \multicolumn{2}{|c|}{Оценка в бальной шкале} & Коэффициент значимости $\alpha_{i}$ \\
  \ & \ & П1 & Разработанная система & \ \\ \hline
  1 & Моделирование стандартного режима & 9 & 8 & 0,3 \\ \hline
  2 & Моделирование критического режима(атака на отказ в обслуживании) & 9 & 8 & 0,4 \\ \hline
  3 & Моделирование атаки на проникновение & 2 & 5 & 0,2 \\ \hline
  4 & Пользовательский интерфейс & 8 & 8 & 0,1 \\ \hline
\end{tabular}
\end{table}

Уровень качества вычисляется по формуле:

\begin{center}
  $K_{\text{кач}} = \sum_{i=1}^n \alpha_{i} \cdot \dfrac{\text{Б}_{i}}{\text{Б}_{\text{баз}}} \approx 1,22$
\end{center}

Расчет эффективности НИР:

\begin{center}
  $\text{Э}_{\text{НИР}} = \dfrac{C_{O}}{K_{\text{кач}} \cdot 100\%} = \dfrac{208121}{122} = 1706  \text{руб}/\%$
\end{center}

Это означает, что для того, чтобы повысить уровень качества на один процент было затрачено 1706 руб.

Кроме того, следует отметить, что разработанная система позволяет уменьшить риски при обработке важной информации, что в конечном итоге приведет к снижению издержек и потерь пользователя.

Все вышеизложенное показывает экономическую целесообразность разработки НТПр.

\subsection*{Выводы}

По результатам проведенной экономической оценки, можно сделать следующие выводы:

\begin{enumerate}
  \item Трудоемкость НТПр составляет 73 чел/дней для инженера и 9чел/дней для старшего научного сотрудника.
  \item Себестоимость НТПр составляет 208121 руб.
  \item Уровень качества НТПр составляет 1,22
\end{enumerate}

К сожалению отсутствие информации не позволяет провести оценку экономической эффективности, но высокий технический уровень решения делает научную разработку эффективной.

