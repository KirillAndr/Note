
    \newpage
    \thispagestyle{empty}
    \ \\
    \ \\
    \ \\
    \ \\
    \ \\
    \ \\
    \ \\
    \ \\
    \ \\

    \begin{center}

        \textbf{ПРОГРАММА ДЛЯ ЭВМ.}

        \large\textbf{Система моделирования сетевых атак.}

        Фрагменты исходного текста программы.
        
        120 стр.
    \end{center}

    \ \\
    \ \\
    \ \\
    \ \\
    \ \\
    \ \\
    \ \\
    \ \\
    \ \\



    Правообладатель: Андрианов Кирилл Сергеевич.

    Автор: Андрианов Кирилл Сергеевич.
    \ \\
    \ \\
    \ \\
    \ \\
    \ \\
    \ \\


    \copyright Андрианов К.С., 2013.


    \ \\
    \ \\
    \ \\


    \begin{center}
        Санкт-Петербург

        2013
    \end{center}

\newpage

    \begin{center}
        \large\textbf{ СОСТАВ ПРОГРАММНОГО КОМПЛЕКСА.}
    \end{center}

    Каталог bin.
    \begin{itemize}
        \item каталог Main
        \begin{itemize}
            \item каталог Devices
            \begin{itemize}
                \item NetworkDeviceConfiguration -- хранение характеристик модели сетевого компонента.
            \end{itemize}
            \item каталог managers
            \begin{itemize}
                \item ApplicationsManager -- хранение характеристик моделей приложений.
                \item ThreadPoolManager -- управление потоками в процессе моделирования.
            \end{itemize}
            \item каталог model
            \begin{itemize}
                        \item каталог scripts -- хранятся объекты, описывающие частное поведение моделей приложений
                        \item Application -- объект, описывающий общее поведение моделей приложений
                        \item LowLevelApplication -- интерфейс, декларирующий методы взаимодействия с моделями протоколов сетевого уровня.
                        \item Vulnerability -- интерфейс, декларирующий методы, используемые в модели уязвимости.
                        \item NetworkConnection -- объект, являющийся моделью физического соединения(Канала связи).
                        \item Transmitter -- объект, являющийся моделью устройства, генерирующего и принимающего сигналы по каналу связи.
                               \item ByteQueue -- очередь байт для отправки или обработки получения.
                               \item ConnectionControlSubSystem -- объект-модель системы контроля соединения в протоколе TCP.
                               \item ConnectionConfiguration -- объект, сохраняющий параметры соединения.
                               \item SegmentControlSubSystem -- объект, реализующий соединение, управляющий передачей пакетов.
                               \item TCPSegment -- объект, описывающий структуру данных, используемую в протоколе TCP.
                        \item DataRepresentationSystem -- объект, преобразующий данные их формы, используемой в приложении в форму, используемую в протоколе.
                        \item TCPSocket -- объект, используемый для обеспечения взаимодействия между приложением и протоколами передачи данных.
                \item Generator -- объект, являющийся моделью для имитации сетевого трафика, создаваемого пользователем.
                \item Network -- объект, являющийся верхним уровнем модели сети и используемый для кофигурирования и управления глобальными параметрами модели.
                \item NetworkDevice -- объект, являющийся моделью устройства, участвующего в сетевом взаимодействии.
            \end{itemize}
            \item каталог protocols
                \begin{itemize}
                    \item каталог scipts -- в нем содержатся файлы, описывающие реализацию конкретного протокола.
                    \item AbstractProtocol -- класс, описывающий общее для всех протоколов операции.
                    \item EnviromentProtocol -- класс описывающий действия для протоколов управления средой передачи данных.
                    \item ProtocolScript -- интерфейс, декларирующий методы, используемые в протоколе.
                    \item ProtocolStack -- объект, объединяющий протоколы, для управления и конфигурации.
                \end{itemize}
            \item каталог subsystems
                \begin{itemize}
                    \item NetworkConfigurationLoader -- объект, который считывает и хранит конфигурацию модели.
                \end{itemize}
        \end{itemize}
        \item Config -- объект, хранящий глобальные переменные для процесса моделирования.
        \item Main -- точка входа в программу и запуск необходимых подсистем.
    \end{itemize}

    Каталог data

    \begin{itemize}
        \item application\_description.xml -- файл, в котором хранятся настройки моделей приложений.
        \item network\_devices\_confguration.xml -- файл, в котором хранятся настройки моделей сетевых устройств.
        \item protocol\_stacks.xml -- файл, в котором хранятся настройки моделей сетевых протоколов.
    \end{itemize}
