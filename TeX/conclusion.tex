\newpage 
\chapter*{Заключение}

   Целью данного дипломного проекта являлась реализация системы моделирования атак на вычислительную сеть. В процессе проектирования было проведено исследование существующих языков имитационного моделирования для использования их в качестве возможных альтернатив созданию новой системы. Описанные средства моделирования являются непригодными для использования в рамках поставленной задачи, так как обладают высоким уровнем абстракции, относительно предметной области, и требуют специальных знаний и навыков в области моделирования, в то время как разрабатываемая система ориентирована на использование инженером по безопасности. 
   
   В связи с этим было решено использовать язык программирования общего назначения JAVA. В связи с разнородностью объектов, входящих в состав предметной области, было использовано несколько подходов к построению моделей, а именно: модель детерминированного конечного автомата, математические модели, описывающие генерацию трафика. Рассматривая предметную область как совокупность независимых компонентов и процесса их взаимодействия, были реализованы отдельные части моделей, которые могут быть использованы пользователем для построения собственных моделей. 
   
   Реализованная система обладает возможностями анализа загруженности сетевых узлов, маршрутов трафика, а так же возможностью проведения тестов на проникновение. Каждая из этих возможностей позволяет моделировать атаки различного рода, оценивать степень защищенности системы и эффективность мер, принимаемых для повышения уровня безопасности. 
   
   Одним из главных требований к разрабатываемой системе была гибкость, то есть возможность использования различных моделей поведения для устройств, изменение и добавление новых моделей без внесения крупных изменений в исходный код приложения. Реализацией этого требования стали классы, описывающие поведение моделей устройств и приложений. Соответствие между устройством и используемым приложением устанавливается в конфигурационном файле и является уникальным для каждого устройства.
   
    Таким образом можно сделать вывод, что поставленные задачи решены полностью. 
    
    Экономическая оценка результатов НТПр, выполненная в ходе проектирования имеет следующие показатели:
    
    \begin{itemize}
        \item Трудоемкость НТПр составляет 73 чел.дней для инженера и 9 чел.дней для руководителя.
        \item Себестоимость НТПр составляет 208121 руб.
        \item Уровень качества НТПр составляет 1.22.
    \end{itemize}

    \subsubsection{Перспективы развития системы}
    
    В ходе дипломного проектирования была создана система, позволяющая проводить эксперименты с моделью локальной сети. Данные, полученные в результате тестов, сохраняются в подсистеме сбора данных. Для принятия решения относительно уровня безопасности системы и мер, которые возможно принять для его повышения, данные, полученные в ходе экспериментов необходимо проанализировать и по результатам анализа вынести соответствующее решение. В связи с этим дальнейшим развитием данной системы, для использования в ходе проектирования локальной сети, является реализация модуля статистического анализа данных и системы поддержки принятия решений, которая генерирует рекомендации по улучшению уровня безопасности локальной сети.
