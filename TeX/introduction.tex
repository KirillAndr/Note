    В современном мире средства вычислительной техники широко используются во многих отраслях деятельности человека. Основной их задачей является обработка и хранение информации поступающей из вне. Такие системы должны обеспечивать доступ к данным и гарантировать их актуальность. Для решения проблемы быстрого получения информации конечным пользователям, отдельные компьютеры объединяются в локальные сети. Подобное объединение можно проследить вплоть до глобальной сети интернет. Однако, решив проблему, связанную с доступностью информации, соединение компьютеров обострило проблемы связанные с безопасностью данных. В случае с локальной машиной не имеющей активного сетевого соединения проблема безопасности решается ограничение физического доступа к терминалу, однако объединив рабочие станции, зона, которую необходимо контролировать, резко возрастает. Кроме того, с внедрением беспроводных технологий, это зона не имеет четких границ, а существующее подключение сети к интернет открывает потенциальную возможность для атаки человеку из любой точки мира. В зависимости от сложности архитектуры, от используемого оборудования и уровня подготовленности сотрудников, имеющих подключение к локальной сети, обнаружение и ликвидация уязвимостей резко возрастает. Принимая во внимание вышеперечисленные факторы, инженеру, ответственному за построение локальной сети, необходимо иметь возможность оценки того или иного архитектурного или программного решения с точки зрения актуальности его применения для конкретной ситуации. Для подобного рода оценок существуют различные методы и средства моделирования.
    
    Применение методов моделирования на этапе проектирования сети позволяет предугадать возможные проблемы при реализации выбранного решения, а также принять меры по обеспечению безопасного функционирования системы. К таким мерам относятся аппаратные средства, ограничивающие доступ к защищаемым ресурсам, протоколы криптографической защиты, программное обеспечение с возможностью мониторинга состояния сети. Различные источники придерживаются разных определений процесса моделирования, так как крайне сложно подобрать такое определение, которое в полной мере охватило деятельность по моделированию. Определение модели по А. А. Ляпунову: Моделирование – это опосредованное практическое или теоретическое исследование объекта, при котором непосредственно изучается не сам интересующий нас объект, а некоторая вспомогательная искусственная или естественная система (модель): 
    
    \begin{itemize}
        \item находящаяся в некотором объективном соответствии с познаваемым объектом;
        \item способная замещать его в определенных отношениях;
        \item дающая при её исследовании, в конечном счете, информацию о самом моделируемом объекте;
    \end{itemize}
    
    Модель (лат. modulus – мера) – это объект заместитель объекта-оригинала, обеспечивающий изучение некоторых свойств оригинала. Замещение одного объекта другим с целью получения информации о важнейших свойствах объекта-оригинала с помощью объекта-модели называется моделированием.Под математическим моделированием будем понимать процесс установления соответствия данному реальному объекту некоторого математического объекта, называемого математической моделью, иисследование этой модели, позволяющее получать характеристики рассматриваемого реального объекта. Вид математической модели зависит как от природы реального объекта, так и задач исследования объекта и требуемой достоверности и точности решения этой задачи.[1] Основная сложность математического моделирования заключается в том, что исследуемые процессы необходимо формализовать математическими функциями, что не всегда возможно. Некоторые процессы, например, действия пользователя в системе, сложно свести к математическим функциям.
    
    Имитационное моделирование – это метод, позволяющий строить модели, описывающие процессы так, как они проходили бы в действительности. Отличительными чертами данного метода является то, что у исследователя есть возможность изменять параметры системы, изучать процесс с одними входными данными или некоторым набором, исследовать развитие процесса во времени. Имитационное моделирование применяется тогда, когда неэффективно с точки зрения затраченных ресурсов и полученных результатов экспериментировать на реальном объекте. Невозможно построить математическую модель в связи с присутствующими в системе нелинейностях, стохастическими переменными или если необходимо исследовать систему во времени. Существующие программные решения позволяют строить модели компьютерных сетей, и исследовать с помощью этой модели различные процессы, происходящие в системе. Однако большинство таких решений созданы для использования системными администраторами, и поэтому позволяют исследовать характеристики системы безотносительно к безопасности. Данные, полученные при использовании такой системы, могут быть абсолютно бесполезными для инженера по безопасности. Существующие решения, отражающие работу сети с точки зрения безопасности являются коммерческими, что затрудняет их использование в образовательных целях или являются свободно распространяемыми, но в этом случае, обычно, обладают малым числом возможностей и неудобным для использования графическим интерфейсом. Таким образом целью данной работы является создание стенда для моделирования работы вычислительной сети с точки зрения информационной безопасности. Данный стенд должен предоставлять возможность построения сетей различных архитектур и топологий, выявлять "узкие"места в локальной сети замедляющие ее работу или представляющие собой уязвимость, а так же изучить состояние и поведение системы при различного рода атаках. В качестве объекта исследования будем рассматривать процесс моделирования взаимодействия различных устройств, объединенных в вычислительную сеть. В рамках данной работы будет построена система, позволяющая моделировать алгоритмы работы устройств на различных уровнях эталонной модели взаимодействия открытых систем(OSI). Протоколы транспортного сетевого и канального уровней возможно моделировать без привлечения без привлечения математических моделей и аппарата математической статистики. В то время как более высокие уровни взаимодействия связаны с действиями пользователя в системе и не могут быть точно воспроизведены. В связи с этим при моделировании генерации трафика в сети, то есть обычного режима работы, будут использованы математические модели, построенные при исследовании данного явления. При моделировании работы протоколов нижнего уровня и физической передачи данных возникаю проблемы иного характера. Они связаны в первую очередь с наличием ошибки в канале связи. Учитывая то, что в реальных условиях эта ошибка носит, преимущественно, случайный характер, моделирование подобных инцидентов будет проводится с использованием методов математической статистки.
    
    Для построения модели вычислительной сети будет применен метод имитационного моделирования. Благодаря этому методу система будет иметь необходимую гибкость при настройке параметров. Наиболее удобным видом моделирования для данной задачи является метод агентного моделирования, так как он позволяет описывать поведение каждого компонента системы в отдельности, а так же порядок их взаимодействия. Описанные преимущества позволяют произвести декомпозицию задачи и построить систему необходимого уровня сложности.
    
    Для построения целевой системы необходимо разработать архитектуру будущего приложения. Она должна включать в себя возможности по построению моделей вычислительных сетей различной конфигурации, предусматривать механизмы настройки большинства компонентов системы, обладать гибкостью для возможного расширения используемого оборудования и протоколов. Следующим этапом разработки является алгоритмизация математических моделей, описывающих генерацию трафика пользователем в вычислительной сети. Результатом данного этапа будет система, в достаточной степени точно отражающая реальные процессы, протекающие в вычислительной сети. Третьим этапом разработки станет подсистема сбора статистической информации по различным устройствам. Данная подсистема занимается сбором данных со всех устройств, входящих в состав моделируемой сети и позволяет проследить динамику изменения состояний различных узлов. Заключительным этапом является создание подсистемы, моделирующей различные виды атак на локальную сеть. В рамках данной подсистемы описываются устройства и программы(алгоритмы), используемые при атаках на вычислительные сети, определяются механизмы подключение к локальной сети, моделируются действия злоумышленника. По последнему пункту будет рассмотрено такое понятие, как мотивация злоумышленника, которая влияет на его заинтересованность в компрометации системы. 
    
    Для решения поставленных задач будет использоваться язык программирования JAVA с использованием различных расширений.


