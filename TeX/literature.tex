\newpage

\chapter*{Список использованной литературы}

\renewcommand{\labelenumi}{\arabic{enumi}.}

\begin{enumerate}
    \item Советов Б.Я., Яковлев С.А., Моделирование систем –М.:Высш.Шк., 2001.

    \item Stefan Karpinski Realistic model of local network traffic, 2005.

    \item Добровольский Е.В., Нечипорук О.Л., Моделирование сетевого трафика с использование контекстных методов, 2005.

    \item  Leland W.E., Taqqu M.S. Willinger W., Wilson D.V. On the self-similar nature of ethernet traffic, 1994.

    \item Таненбаум Э., Уэзеролл Д., Компьютерные сети –Спб.:Питер, 2012.

    \item  Gorodetsk V., Kotenko I., Attacks against computer network: Formal grammar-based framework and simulation tool, 2002.

    \item  Lye K., Wing J. Game strategies in network security, 2005.

    \item  Kumar S., Spafford E.H. An application of pattern matching in intrusion detection, 1994.

    \item Гудов А.М., Семехина М.В., Имитационное моделирование процессов передачи трафика в вычислительных сетях.

    \item Шенон Р., Имитационное моделирование систем. Искусство и наука.

    \item Степашкин М.В., Котенко И.В., Богданов В.С., Моделирование атак для активного анализа уязвимостей компьютерных сетей.

    \item  Ingols K., Chu M.,Lippmann R., Webster S., Boyer S.,  Modeling modern network attacks and and contermeasures using attack graphs, 2009.

    \item  Kotenko I., Stepashkin M., Ulanov A. Agent-based modeling and simulation of malefactors attacks against computer networks, 2006.

    \item  Lathrop S. D., Hill J., Surdu J.R.,  modeling network attacks.

    \item  Sarraute C., Fernando Miranda, Jose I. Orlicki, Simulation of computer networks.

    \item  Goldman R.P., Stochastic model for intrusions, 2002.

    \item  Liu P., Zang W., Incentive-based modeling and inference if attacker intent, objectives, and strategies, 2005

    \item  Cohen F. Simulation cyber attacks, defences, and consequences, 1999.

    \item  Dawkins J., Campbell C., Hall J. Modeling network attacks: Extending the attack tree paradigm, 2002.

    \item  Ingols K., Lippmann R., Piwowarski K., Practical attack graph generation for network defence, 2006.

\end{enumerate} 