\newpage
\chapter{Охрана интеллектуальной собственности}

    В ходе выполнения дипломного проекта мною, Андриановым Кириллом Сергеевичем, была разработана программа для ЭВМ "Система моделирования сетевых атак". Этот результат научно-технической деятельности входит в перечень охраняемых объектов интеллектуальной собственности Гражданского кодекса РФ. Программа разработана по личной инициативе , права на программу принадлежат автору - Андрианову Кириллу Сергеевичу, так как программа разработана в рамках учебного процесса.

    Под интеллектуальной собственностью понимают особый вид гражданских прав (исключительное право) в отношении результатов интеллектуальной деятельности, таких как изобретения, промышленные образцы (дизайн), компьютерные программы, другие произведения науки, произведения литературы, искусства, которые принято называть объектами интеллектуальной собственности, а также различных средств индивидуализации производителя товаров и услуг, таких как товарные знаки, знаки обслуживания, фирменные наименования и др. [2, ст. 1225]. Основным содержанием таких прав является монополия их владельца на использование этих объектов, включая право запретить или разрешить их использование другим, а также право переуступить другому лицу эти правомочия или отказаться от них вовсе.

    Согласно определению интеллектуальной собственности, принятому в российском законодательстве, а также на основании определения Стокгольмской конференции от 14 июля 1967 г., программы для ЭВМ (компьютерные программы) и базы данных относятся к объектам интеллектуальной собственности. Программам для ЭВМ и базам данных предоставляется охрана нормами авторского права как литературным произведениям в соответствии с Бернской конвенцией, причем программы для ЭВМ \
    охраняются как литературные произведения, а базы данных - как сборники.

    В Российской Федерации вопросы предоставления правовой охраны программам для ЭВМ и базам данных регулируются Гражданским кодексом РФ, Часть 4 (ГК РФ Ч.4).

    Под программой для ЭВМ понимается "... представленная в объективной форме совокупность данных и команд, предназначенных для функционирования ЭВМ и других компьютерных устройств в целях получения определенного результата". Кроме того, в понятие программы для ЭВМ входят "...подготовительные материалы, полученные в ходе разработки программы для ЭВМ, и порождаемые ею аудиовизуальные отображения" [2, ст. 1261].

    С точки зрения программистов и пользователей программа для ЭВМ представляет собой детализацию алгоритма решения какой-либо задачи и выражена в форме определенной последовательности предписаний, обеспечивающих выполнение компьютером преобразования исходных данных в искомый результат. Можно выделить следующие объективные формы представления программы для ЭВМ:

    \begin{itemize}
        \item исходная программа (или исходный текст) - последовательность предписаний на алгоритмическом (понятном человеку) языке высокого уровня, предназначенных для автоматизированного перевода этих предписаний в последовательность команд в объектном коде;
        \item рабочая программа (или объектный код) - последовательность машинных команд, т. е. команд, представленных на языке, понятном ЭВМ;
        \item программа, временно введенная в память ЭВМ, - совокупность физических состояний элементов памяти запоминающего устройства ЭВМ (ОЗУ), сохраняющихся до прекращения подачи электропитания к ЭВМ;
        \item программа, постоянно хранимая в памяти ЭВМ, - представленная на языке машины команда (или серия команд), выполненная в виде физических особенностей участка интегральной схемы, сохраняющихся независимо от подачи электропитания.
    \end{itemize}

    Исходная и рабочая программы, как правило, представляются в виде записи на том или ином языке, выполненной на бумаге или машиночитаемом носителе данных: магнитном или оптическом диске, магнитной ленте и т. п.

    Предоставляемая законодательством правовая охрана распространяется "... на все виды программ для ЭВМ (в том числе на операционные системы и программные комплексы), которые могут быть выражены на любом языке и в любой форме, включая исходный текст и объектный код …" [2, ст.1261]. Так как преобразование исходного текста программы для ЭВМ в объектный (машинный) код с помощью специальных программ-трансляторов не меняет сущности данной программы как произведения, то если охраняется исходный текст программы, значит, охране подлежит и соответствующий ей объектный код. Обратное тоже справедливо.

    Правовая охрана программ для ЭВМ распространяется только в отношении формы их выражения и "… не распространяется на идеи, концепции, принципы, методы, процессы, системы, способы, решения технических, организационных или иных задач, открытия, факты, языки программирования" [2, ст.1259, п. 5].

    Предпосылкой охраноспособности программы для ЭВМ и базы данных является их творческий характер, т. е. они должны быть продуктом личного творчества автора. Творческий характер деятельности автора предполагается до тех пор, пока не доказано обратное [2, ст. 1257].

    Момент возникновения авторского права является важнейшим юридическим фактом, который устанавливается в силу создания произведения (программы для ЭВМ или базы данных). "Для возникновения, осуществления и защиты авторских прав не требуется регистрация произведения или соблюдение каких-либо иных формальностей" [2, ст.1259, п.4].

    Часто возникает вопрос: насколько необходимо для возникновения прав на программу для ЭВМ или базу данных их обнародование? Закон устанавливает, что это не является обязательным условием: "Авторские права распространяется как на обнародованные, так и на необнародованные произведения, выраженные в какой-либо объективной форме …" [2, ст. 1259, п. 3].

    Таким образом, только сам факт создания программы или базы данных, зафиксированных в объективной форме, является основанием возникновения авторского права на эти объекты. С этого момента права автора или иного правообладателя защищаются законом.

    Права в отношении программ для ЭВМ и баз данных подразделяются на личные неимущественные и исключительные права.

    Личные права включают право авторства, право на имя и право на неприкосновенность (целостность), право на обнародование программы для ЭВМ или базы данных [2, ст. 1265-1268]. Они связаны непосредственно с автором программы для ЭВМ или базы данных: принадлежат лицу, чьим творческим трудом созданы программа для ЭВМ или база данных - автору, являются неотчуждаемыми, т. е. не могут быть переуступлены другому лицу, и не ограничены каким-либо сроком [2, ст. 1228].

    Исключительные права непосредственно связаны с понятием "использования" программ для ЭВМ и баз данных: "Автору произведения или иному правообладателю принадлежит исключительное право использовать произведение … в любой форме и любым не противоречащим закону способом …" [2, ст. 1270, п.1]. При этом под использованием понимается осуществление определенных действий с программами для ЭВМ или базами данных, а именно: опубликование (выпуск в свет); воспроизведение (полное или частичное) в любой форме, любыми способами; распространение; модификацию и иное использование [2, ст. 1270, п.2]. Они могут принадлежать автору или иному правообладателю (гражданину или юридическому лицу). Правообладатель может распоряжаться исключительным правом на произведение [2, ст. 1270, п.1], в том числе по своему усмотрению разрешать или запрещать другим лицам использование [2, ст. 1229, п.1]. Распоряжение принадлежащим правообладателю исключительным правом может осуществляться любым, не противоречащим закону и существу такого исключительного права способом, в том числе путем его отчуждения по договору другому лицу или предоставления другому лицу права использования [2, ст. 1233, п.1]. Срок действия исключительного права ограничен

    Каждая составляющая понятия использования программы для ЭВМ или базы данных имеет конкретное содержание, которое также определено законом:
    \begin{itemize}
        \item   воспроизведение - "... изготовление одного или более экземпляров произведения или его части в любой материальной форме, … в том числе запись в память ЭВМ" [2, ст. 1270, п. 2, п.п.1];
        \item распространение - предоставление доступа к произведению "... путем продажи или иного отчуждения его оригинала или экземпляров" [2, ст. 1270, п. 2, п.п.2].
        \item публичный показ (выпуск в свет) - "... любая демонстрация оригинала или экземпляров произведения непосредственно … … либо с помощью технических средств в месте, открытом для свободного посещения, или в месте, где присутствует значительное число лиц … " [2, ст. 1270, п. 2, п.п.3];
    \end{itemize}

    Обращает на себя внимание то, что понятие "использование" не связано с функционированием программы (или ее выполнением) с целью получения результата. Последнее лучше относить к понятию "потребление" или "пользование" программы. Поэтому каждый раз, когда пользователь запускает программу для того, чтобы произвести расчеты, построить графики или таблицы и т. п., он не "использует" (поскольку не создается новый экземпляр), а "потребляет" программу, не нарушая при этом ничьих прав.

    В целях оповещения о своих правах правообладатель "... вправе использовать знак охраны авторского права, который помещается на каждом экземпляре произведения и состоит из следующих элементов: латинской буквы С в окружности; имени или наименования правообладателя; года первого опубликования произведения" [2, ст. 1271]. Знак охраны авторского права может проставляться на упаковке, на самом программном продукте, а также на всех сопроводительных и дополнительных материалах, однако это не является обязательным. Следует иметь в виду, что сам знак ничего не защищает (защищает закон!), и что наличие или отсутствие знака охраны никак не связано с возникновением авторского права. Знак охраны авторского права - это цивилизованная форма предупреждения желающих использовать данный объект, что права на него охраняются законом, с одновременным сообщением о том, кому эти права принадлежат.

    Исключительные права на программу или базу данных переходят по наследству в установленном законом порядке, и их можно реализовать в течение срока действия авторского права.

    Права на программу для ЭВМ или базу данных не связаны с правом собственности на материальный носитель, на котором они зафиксированы. Передача прав на материальный носитель не влечет за собой передачи каких-либо прав на программу для ЭВМ или базу данных [2, ст.1227]. Иными словами, передача носителя информации (например, дискеты) с зафиксированной на нем программой третьему лицу не означает передачи каких-либо прав на эту программу.

    
    \newpage
    \thispagestyle{empty}
    \ \\
    \ \\
    \ \\
    \ \\
    \ \\
    \ \\
    \ \\
    \ \\
    \ \\

    \begin{center}

        \textbf{ПРОГРАММА ДЛЯ ЭВМ.}

        \large\textbf{Система моделирования сетевых атак.}

        Фрагменты исходного текста программы.
        
        120 стр.
    \end{center}

    \ \\
    \ \\
    \ \\
    \ \\
    \ \\
    \ \\
    \ \\
    \ \\
    \ \\



    Правообладатель: Андрианов Кирилл Сергеевич.

    Автор: Андрианов Кирилл Сергеевич.
    \ \\
    \ \\
    \ \\
    \ \\
    \ \\
    \ \\


    \copyright Андрианов К.С., 2013.


    \ \\
    \ \\
    \ \\


    \begin{center}
        Санкт-Петербург

        2013
    \end{center}

\newpage

    \begin{center}
        \large\textbf{ СОСТАВ ПРОГРАММНОГО КОМПЛЕКСА.}
    \end{center}

    Каталог bin.
    \begin{itemize}
        \item каталог Main
        \begin{itemize}
            \item каталог Devices
            \begin{itemize}
                \item NetworkDeviceConfiguration -- хранение характеристик модели сетевого компонента.
            \end{itemize}
            \item каталог managers
            \begin{itemize}
                \item ApplicationsManager -- хранение характеристик моделей приложений.
                \item ThreadPoolManager -- управление потоками в процессе моделирования.
            \end{itemize}
            \item каталог model
            \begin{itemize}
                        \item каталог scripts -- хранятся объекты, описывающие частное поведение моделей приложений
                        \item Application -- объект, описывающий общее поведение моделей приложений
                        \item LowLevelApplication -- интерфейс, декларирующий методы взаимодействия с моделями протоколов сетевого уровня.
                        \item Vulnerability -- интерфейс, декларирующий методы, используемые в модели уязвимости.
                        \item NetworkConnection -- объект, являющийся моделью физического соединения(Канала связи).
                        \item Transmitter -- объект, являющийся моделью устройства, генерирующего и принимающего сигналы по каналу связи.
                               \item ByteQueue -- очередь байт для отправки или обработки получения.
                               \item ConnectionControlSubSystem -- объект-модель системы контроля соединения в протоколе TCP.
                               \item ConnectionConfiguration -- объект, сохраняющий параметры соединения.
                               \item SegmentControlSubSystem -- объект, реализующий соединение, управляющий передачей пакетов.
                               \item TCPSegment -- объект, описывающий структуру данных, используемую в протоколе TCP.
                        \item DataRepresentationSystem -- объект, преобразующий данные их формы, используемой в приложении в форму, используемую в протоколе.
                        \item TCPSocket -- объект, используемый для обеспечения взаимодействия между приложением и протоколами передачи данных.
                \item Generator -- объект, являющийся моделью для имитации сетевого трафика, создаваемого пользователем.
                \item Network -- объект, являющийся верхним уровнем модели сети и используемый для кофигурирования и управления глобальными параметрами модели.
                \item NetworkDevice -- объект, являющийся моделью устройства, участвующего в сетевом взаимодействии.
            \end{itemize}
            \item каталог protocols
                \begin{itemize}
                    \item каталог scipts -- в нем содержатся файлы, описывающие реализацию конкретного протокола.
                    \item AbstractProtocol -- класс, описывающий общее для всех протоколов операции.
                    \item EnviromentProtocol -- класс описывающий действия для протоколов управления средой передачи данных.
                    \item ProtocolScript -- интерфейс, декларирующий методы, используемые в протоколе.
                    \item ProtocolStack -- объект, объединяющий протоколы, для управления и конфигурации.
                \end{itemize}
            \item каталог subsystems
                \begin{itemize}
                    \item NetworkConfigurationLoader -- объект, который считывает и хранит конфигурацию модели.
                \end{itemize}
        \end{itemize}
        \item Config -- объект, хранящий глобальные переменные для процесса моделирования.
        \item Main -- точка входа в программу и запуск необходимых подсистем.
    \end{itemize}

    Каталог data

    \begin{itemize}
        \item application\_description.xml -- файл, в котором хранятся настройки моделей приложений.
        \item network\_devices\_confguration.xml -- файл, в котором хранятся настройки моделей сетевых устройств.
        \item protocol\_stacks.xml -- файл, в котором хранятся настройки моделей сетевых протоколов.
    \end{itemize}


    \subsubsection{Реферат.}

    \begin{tabular}{p{15em}p{25em}}
        Автор: & Андрианов Кирилл Сергеевич \\
        Правобладатель: & Андрианов Кирилл Сергеевич \\
        Программа для ЭВМ: & Система моделирования сетевых атак \\
        Аннотация : & Система предназначена для моделирования процесса атак на вычислительную сеть. Система реализована таким образом, что каждый ее отдельный компонент может быть модернизирован без ущерба для остальных частей программы. Модернизация может включать в себя как изменение работы алгоритма, уже существующего, так и создание нового, описывающего иную модель поведения объекта, отличную от рассмотренных. Программа позволяет строить модели сетей различной структуры и оценивать их работоспособность и эффективность в различных условиях. Так же в программе существует возможность моделирования атаки на отдельные рабочие станции. Данная возможность осуществляется с помощью моделей приложений и моделей уязвимостей. Эта модель представляет собой вероятностный эксперимент, результатом которого является успех или неудача. \\

        Тип ЭВМ: & IBM PC-совместимый компьютер. \\
        ОС : & Любая. Необходима JAVA-машина версии не ниже 6.\\
        Язык программирования : & JAVA. \\
        Объем : & 246 Кбайт.
    \end{tabular}

\newpage

\begin{center}
    ЛИЦЕНЗИОННЫЙ ДОГОВОР

    НА ИСПОЛЬЗОВАНИЕ ПРОГРАММЫ ДЛЯ ЭВМ
\end{center}

    Стороны в Договоре:

    Гражданин Андрианов Кирилл Сергеевич, проживающий по адресу: 198320, г. Санкт-Петербург, Гатчинское шоссе, д. 13, корп.1 кв. 6, именуемый в дальнейшем "ЛИЦЕНЗИАР", с одной стороны, и Государственное образовательное учреждение высшего профессионально-го образования "Санкт-Петербургский государственный электротехнический университет "ЛЭТИ" им. В.И.Ульянова (Ленина)", именуемый в дальнейшем "ЛИЦЕНЗИАТ", в лице проректора по научной работе Шестопалова М.Ю., действующего на основании Доверенности, с другой стороны, принимая во внимание:
\begin{enumerate}
    \item что Лицензиар является автором и правообладателем программы для ЭВМ "Система моделирования сетевых атак";
    \item Лицензиат желает получить на условиях настоящего Договора лицензию на ис-пользование упомянутой программы для ЭВМ с целью проведения научных ис-следований в области медицинского приборостроения;
    \item Лицензиар готов предоставить Лицензиату такую лицензию, договорились о следующем.
\end{enumerate}

\begin{enumerate}
    \item Термины и их определения
    \begin{enumerate}
        \item "ПРОГРАММА ДЛЯ ЭВМ (ПрЭВМ)" - программное обеспечение "Система моделирования сетевых атак".
        \item "ДОКУМЕНТАЦИЯ" - комплект документов, передаваемых Лицензиаром Лицензиату, включающий руководство пользователя по применению и обслуживанию программы для ЭВМ.
        \item "ПРОИЗВОДСТВЕННАЯ ПЛОЩАДКА" - научные лаборатории и кафедры Лицензиата.
        \item "РАБОЧЕЕ МЕСТО" - конкретная ЭВМ, на которой используется Программа для ЭВМ.
    \end{enumerate}
    \item Предмет Договора
        \begin{enumerate}
            \item Лицензиар предоставляет Лицензиату на срок действия настоящего Договора и за вознаграждение, уплачиваемое Лицензиатом, неисключительную лицензию на использование ПрЭВМ. При этом Лицензиату предоставляется право на установку ПрЭВМ не более чем на 10 (десяти) Рабочих местах.
            \item Лицензиар передает Лицензиату Документацию к ПрЭВМ.
            \item Предоставленное Лицензиату в рамках настоящего Договора право ограничено Производственной площадкой.
            \item Лицензиар осуществляет авторский контроль за соблюдением объемов ис-пользования ПрЭВМ по настоящему Договору, при этом Лицензиат обеспечивает возможность такого контроля.
            \item Лицензиар сохраняет за собой право самому использовать ПрЭВМ и предоставлять неисключительные лицензии на право ее использования третьим лицам.
        \end{enumerate}
    \item Обеспечение Договора
    \begin{enumerate}
        \item Лицензиар передает Лицензиату ПрЭВМ в объеме и виде, достаточном для ее использования, и Документацию в течение 15 (пятнадцати) дней со дня подписания настоящего Договора. ПрЭВМ передается Лицензиату в виде   в количестве 5 (пяти) штук, содержащих ПрЭВМ. По факту передачи ПрЭВМ и Документации составляется акт сдачи-приемки с перечнем переданных материалов, подписываемый обеими Сторонами.
        \item Если Лицензиат установит неполноту или неправильность полученных ПрЭВМ или Документации, то Лицензиар в течение 15 (пятнадцати) дней после сообщения ему об этом Лицензиатом обязан передать недостающие материалы или устранить недостатки ранее переданных ПрЭВМ и Документации.
        \item Для оказания помощи в освоении ПрЭВМ Лицензиар по просьбе Лицензиата оказывает консультации пользователям ПрЭВМ.
        \item Для целей использования ПрЭВМ в объеме, предусмотренном п. 2.1 настоящего Договора, Лицензиат может изготавливать в необходимом ему количестве копии ПрЭВМ и копии Документации.
    \end{enumerate}
    \item Усовершенствования.
    \begin{enumerate}
        \item Лицензиар обязуется незамедлительно информировать Лицензиата о всех произведенных им усовершенствованиях ПрЭВМ и, при желании Лицензиата, передать ему в согласованные сроки новые варианты ПрЭВМ. В отношении новых вариантов ПрЭВМ, переданных Лицензиаром Лицензиату, распространяются все условия настоящего Договора.
        \item Лицензиат обязуется предоставлять Лицензиару информацию об использовании ПрЭВМ, которая могла бы быть полезной для усовершенствования ПрЭВМ.
    \end{enumerate}
    \item Платежи.
    \begin{enumerate}
        \item За предоставление прав, предусмотренных настоящим Договором, Лицензиат выплачивает Лицензиару единовременное вознаграждение в размере 20000(двадцать тысяч) рублей.
        \item Вознаграждение, предусмотренное п. 5.1 настоящего Договора, выплачивается Лицензиатом в течение 30 (тридцати) дней, следующих после подписания акта приемки-сдачи.
    \end{enumerate}
    \item Реклама
    \begin{enumerate}
        \item Лицензиат обязуется при опубликовании результатов исследований, полученных с использованием ПрЭВМ, сообщать в рекламных целях, что исследования производились с использованием ПрЭВМ Лицензиара с указанием авторского права Лицензиара.
    \end{enumerate}
    \item Защита передаваемых прав
    \begin{enumerate}
        \item Лицензиат обязуется не вносить самовольно каких-либо изменений в ПрЭВМ и Документацию и не дополнять их какими-либо комментариями. Подобные изменения или дополнения возможны только с согласия Лицензиара.
        \item Лицензиат обязуется предпринимать все необходимые меры для предотвращения несанкционированного копирования ПрЭВМ и Документации третьими лицами, а также несанкционированной передачи ПрЭВМ и Документации работниками Лицензиата третьим лицам.
        \item Если Лицензиату станет известно о противоправном использовании ПрЭВМ третьими лицами, то он незамедлительно сообщит об этом Лицензиару.
    \end{enumerate}
    \item Ответственность Сторон и разрешение споров.
    \begin{enumerate}
        \item За невыполнение или ненадлежащее выполнение обязательств по настоящему Договору Стороны несут имущественную ответственность в соответствии с действующим законодательством.
        \item Стороны освобождаются от ответственности за неисполнение или ненадлежащее исполнение обязательств, принятых по настоящему Договору, если неисполнение явилось следствием обстоятельств непреодолимой силы (форс-мажор).
        \item Сторона, нарушившая свои обязательства по настоящему Договору, освобождается от ответственности за неисполнение или ненадлежащее исполнение этих обязательств, если это нарушение было вызвано причинами, за которые отвечает другая Сторона.
        \item В случае возникновения споров между Лицензиаром и Лицензиатом по вопросам, предусмотренным настоящим Договором, Стороны примут все меры к разрешению их путем переговоров между собой. В случае невозможности разрешения указанных споров путем переговоров они будут разрешаться в порядке, предусмотренном действующим законодательством.
    \end{enumerate}
    \item Срок действия Договора и условия его расторжения
    \begin{enumerate}
        \item Настоящий Договор заключен на срок 2 года и вступает в силу с даты его подписания обеими Сторонами.
        \item По истечении срока действия настоящего Договора Лицензиат вправе использовать ПрЭВМ, включая усовершенствованные варианты, на Производственной площадке на любом количестве Рабочих мест. При этом обязательства Лицензиата, предусмотренные пп. 7.1 и 7.2 настоящего Договора, сохраняются бессрочно.
        \item Действие настоящего Договора по обоюдному согласию Сторон может быть досрочно прекращено, но не ранее чем через три месяца после предложения об этом одной из Сторон. При этом Лицензиат не освобождается от обязательств по платежам, возникшим до расторжения настоящего Договора.
        \item Настоящий Договор может быть досрочно расторгнут в одностороннем порядке со стороны Лицензиара из-за невыполнения Лицензиатом своих обязательств по пп. 7.1 или 7.2. В этом случае Лицензиат лишается права дальнейшего использования ПрЭВМ в любой форме и обязан вернуть ее Лицензиару.
        \item Если Лицензиат откажется от дальнейшего использования ПрЭВМ, то он уничтожит все имеющиеся у него копии ПрЭВМ.
    \end{enumerate}
    \item Заключительные положения.
    \begin{enumerate}
        \item Все изменения и дополнения к настоящему Договору действительны только в тех случаях, если они совершены в письменной форме и подписаны обеими Сторонами.
        \item Стороны не имеют права передавать свои права и обязательства по на-стоящему Договору третьим лицам без письменного согласия на то другой Стороны.
        \item Во всем остальном, что не предусмотрено условиями настоящего Договора, будут применяться нормы законодательства Российской Федерации.
    \end{enumerate}
    \item Адреса Сторон
    \begin{enumerate}
        \item ЛИЦЕНЗИАР: Андрианов Кирилл   Сергеевич, адрес: 198320, Санкт-Петербург, Гатчинское шоссе д.13,  корп.1, кв.6.
        \item ЛИЦЕНЗИАТ: СПбГЭТУ, адрес: 197376, Санкт-Петербург, ул. Проф. Попова, д. 5.
    \end{enumerate}
\end{enumerate}

Настоящий Договор составлен в двух экземплярах для каждой из Сторон и подписан "\underline{\ \ \ }"\underline{\ \ \ \ \ \ \ \ \ \ } 200\underline{\ \ } г. в г. Санкт-Петербурге.
\\
\\

\begin{tabular}{p{20em}p{20em}}
    ЛИЦЕНЗИАР: &   От ЛИЦЕНЗИАТА: \\
     & Проректор по научной работе СПбГЭТУ \\
    \underline{\ \ \ \ \ \ \ \ \ \ \ }  К.С. Андрианов & 	

            \underline{\ \ \ \ \ \ \ \ \ \ \ \ } В.М. Кутузов
            \\
\end{tabular}
